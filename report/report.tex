\documentclass{article}

% if you need to pass options to natbib, use, e.g.:
%     \PassOptionsToPackage{numbers, compress}{natbib}
% before loading neurips_2025

% The authors should use one of these tracks.
% Before accepting by the NeurIPS conference, select one of the options below.
% 0. "default" for submission
\usepackage[main, final, nonatbib]{neurips_2025}
% the "default" option is equal to the "main" option, which is used for the Main Track with double-blind reviewing.
% 1. "main" option is used for the Main Track
%  \usepackage[main]{neurips_2025}
% 2. "position" option is used for the Position Paper Track
%  \usepackage[position]{neurips_2025}
% 3. "dandb" option is used for the Datasets & Benchmarks Track
 % \usepackage[dandb]{neurips_2025}
% 4. "creativeai" option is used for the Creative AI Track
%  \usepackage[creativeai]{neurips_2025}
% 5. "sglblindworkshop" option is used for the Workshop with single-blind reviewing
 % \usepackage[sglblindworkshop]{neurips_2025}
% 6. "dblblindworkshop" option is used for the Workshop with double-blind reviewing
%  \usepackage[dblblindworkshop]{neurips_2025}

% After being accepted, the authors should add "final" behind the track to compile a camera-ready version.
% 1. Main Track
 % \usepackage[main, final]{neurips_2025}
% 2. Position Paper Track
%  \usepackage[position, final]{neurips_2025}
% 3. Datasets & Benchmarks Track
 % \usepackage[dandb, final]{neurips_2025}
% 4. Creative AI Track
%  \usepackage[creativeai, final]{neurips_2025}
% 5. Workshop with single-blind reviewing
%  \usepackage[sglblindworkshop, final]{neurips_2025}
% 6. Workshop with double-blind reviewing
%  \usepackage[dblblindworkshop, final]{neurips_2025}
% Note. For the workshop paper template, both \title{} and \workshoptitle{} are required, with the former indicating the paper title shown in the title and the latter indicating the workshop title displayed in the footnote.
% For workshops (5., 6.), the authors should add the name of the workshop, "\workshoptitle" command is used to set the workshop title.
% \workshoptitle{WORKSHOP TITLE}

% "preprint" option is used for arXiv or other preprint submissions
 % \usepackage[preprint]{neurips_2025}

% to avoid loading the natbib package, add option nonatbib:
%    \usepackage[nonatbib]{neurips_2025}

\usepackage[utf8]{inputenc} % allow utf-8 input
\usepackage[T1]{fontenc}    % use 8-bit T1 fonts
\usepackage{hyperref}       % hyperlinks
\usepackage{url}            % simple URL typesetting
\usepackage{booktabs}       % professional-quality tables
\usepackage{amsfonts}       % blackboard math symbols
\usepackage{nicefrac}       % compact symbols for 1/2, etc.
\usepackage{microtype}      % microtypography
\usepackage{xcolor}         % colors

\usepackage[numbers]{natbib}
\usepackage{listings}
\usepackage{caption}
\usepackage{csvsimple}
\usepackage{adjustbox}
\usepackage{tcolorbox} % Add in preamble if not already included
\tcbuselibrary{breakable}
\usepackage{float}

\usepackage{tikz}
\usetikzlibrary{shadings}

\lstdefinelanguage{json}{
    basicstyle=\ttfamily\small,
    numbers=none,
    numberstyle=\tiny\color{gray},
    stepnumber=1,
    numbersep=5pt,
    showstringspaces=false,
    breaklines=true,
    frame=single,                   % <-- adds a single line frame
    rulecolor=\color{gray!40},     % <-- light gray frame color
    backgroundcolor=\color{white},
    literate=
     *{0}{{{\color[HTML]{3C78D8}0}}}{1}
      {1}{{{\color[HTML]{3C78D8}1}}}{1}
      {2}{{{\color[HTML]{3C78D8}2}}}{1}
      {3}{{{\color[HTML]{3C78D8}3}}}{1}
      {4}{{{\color[HTML]{3C78D8}4}}}{1}
      {5}{{{\color[HTML]{3C78D8}5}}}{1}
      {6}{{{\color[HTML]{3C78D8}6}}}{1}
      {7}{{{\color[HTML]{3C78D8}7}}}{1}
      {8}{{{\color[HTML]{3C78D8}8}}}{1}
      {9}{{{\color[HTML]{3C78D8}9}}}{1}
      {:}{{{\color[HTML]{666666}{:}}}}{1}
      {,}{{{\color[HTML]{666666}{,}}}}{1}
      {\{}{{{\color[HTML]{AAAAAA}{\{}}}}{1}
      {\}}{{{\color[HTML]{AAAAAA}{\}}}}}{1}
      {[}{{{\color[HTML]{AAAAAA}{[}}}}{1}
      {]}{{{\color[HTML]{AAAAAA}{]}}}}{1},
    stringstyle=\color[HTML]{CE9178},
    keywordstyle=\color[HTML]{569CD6}\bfseries,
    morestring=[b]",
    morekeywords={true,false,null},
    commentstyle=\color[HTML]{6A9955},
}


% Custom command to create gradient-colored PromptGen
\newcommand{\PromptGen}{
    {\bfseries PromptGen}
}

% Bibliography
\begin{filecontents}{references.bib}
@inproceedings{Lee_2025, series={CHI ’25},
title={VeriPlan: Integrating Formal Verification and LLMs into End-User Planning},
url={http://dx.doi.org/10.1145/3706598.3714113},
DOI={10.1145/3706598.3714113},
booktitle={Proceedings of the 2025 CHI Conference on Human Factors in Computing Systems},
publisher={ACM},
author={Lee, Christine P. and Porfirio, David and Wang, Xinyu Jessica and Zhao, Kevin Chenkai and Mutlu, Bilge},
year={2025},
month=apr, pages={1–19},
collection={CHI ’25} }
\end{filecontents}

% Note. For the workshop paper template, both \title{} and \workshoptitle{} are required, with the former indicating the paper title shown in the title and the latter indicating the workshop title displayed in the footnote. 
\title{PromptGen: Dataset Compiler for LLM Verification}


% The \author macro works with any number of authors. There are two commands
% used to separate the names and addresses of multiple authors: \And and \AND.
%
% Using \And between authors leaves it to LaTeX to determine where to break the
% lines. Using \AND forces a line break at that point. So, if LaTeX puts 3 of 4
% authors names on the first line, and the last on the second line, try using
% \AND instead of \And before the third author name.


\author{ 
    Divyansh Rajesh Jain \\
    University of California, Davis \\
    Davis, CA 95616, USA \\
    \texttt{drajeshjain@ucdavis.edu} \\
    \And
    Varsha Sivaprakash\\
    University of California, Davis \\
    Davis, CA 95616, USA \\
    \texttt{varsivaprakash@ucdavis.edu} \\
  % \AND
  % Coauthor \\
  % Affiliation \\
  % Address \\
  % \texttt{email} \\
  % \And
  % Coauthor \\
  % Affiliation \\
  % Address \\
  % \texttt{email} \\
  % \And
  % Coauthor \\
  % Affiliation \\
  % Address \\
  % \texttt{email} \\
}


\begin{document}


\maketitle


\begin{abstract}
  Large Language Models (LLMs) have demonstrated remarkable capabilities across a wide range of tasks, 
  yet they continue to exhibit unpredictable behavior and occasional factual inaccuracies. 
  A persistent challenge in their deployment is the lack of systematic methods for evaluating model 
  performance within specific domains. We introduce \PromptGen, a framework that empowers domain experts 
  to generate high-quality, structured evaluation datasets by encoding domain knowledge and constraints 
  using logic programming. \PromptGen leverages Prolog to represent domain-specific facts and relationships, 
  augmented with natural language annotations that define how these logical structures should be verbalized. 
  Through unification, the system derives all valid instantiations of a domain’s logic and automatically converts 
  them into well-formed English question-answer pairs. This approach provides a scalable, declarative alternative 
  to manual prompt crafting, facilitating the evaluation, fine-tuning, and repair of LLMs in a domain-aware manner. 
  \PromptGen bridges the gap between symbolic reasoning and natural language generation, offering a principled method 
  for aligning LLM evaluation with structured domain expertise.
  The code for this project can be found at \href{https://github.com/svarshh/189g_final_project}{our GitHub repository}.

\end{abstract}

\section{Introduction}
Despite the impressive performance of modern large language models (LLMs), 
they remain prone to generating incorrect or inconsistent outputs in 
unpredictable ways. A core challenge lies in reliably evaluating whether 
an LLM performs well on a specific task or within a particular domain. 
To address this, we propose a method that allows domain experts to formally 
specify the constraints inherent to their domain, enabling the automated 
generation of evaluation datasets that can be used to evaluate, or fine-tune LLMs.

The central research question we explore is whether it is possible to design 
a system that enables users to define domain-specific constraints in a structured, 
logic-based format, such that the system can automatically generate well-formed 
English question-answer (QA) pairs. These QA pairs serve as targeted benchmarks 
for evaluating an LLM’s capabilities in that domain.

The key insight behind our system, \PromptGen, is to leverage Prolog as the underlying 
representation language for both domain constraints and associated natural language 
annotations. \PromptGen uses Prolog’s unification mechanism to exhaustively derive valid 
facts from the logic program and then applies user-provided annotations to convert these 
facts into fluent and grammatically correct QA pairs. This approach represents a 
significant departure from the standard practice of manually writing evaluation prompts. 
Rather than composing prompts by hand, users encode their domain knowledge through an 
enriched logic program. \PromptGen then synthesizes diverse and high-coverage QA pairs 
directly from this structured representation, offering a principled and scalable 
alternative for generating evaluation datasets tailored to specific domains

\section{Related Works}

\paragraph{VeriPlan \cite{Lee_2025}} is a self-verifying system that enables large language models (LLMs) 
to validate and correct their outputs for planning tasks by formally encoding constraints 
derived from user prompts. It employs an LLM-based mapping agent to extract logical and 
temporal constraints, incorporates a mechanism for adjusting constraint flexibility, 
and iteratively refines outputs based on constraint violations. If the generated output 
fails to satisfy the constraints, the violations are fed back to the model in successive 
rounds of refinement; users may also intervene by modifying constraint definitions or 
their strictness. However, VeriPlan is limited in scope, focusing primarily on a narrow 
class of temporal scheduling prompts and requiring human oversight to ensure constraint 
satisfaction. In contrast, PromptGen aims to generalize across a broader range of domains 
by allowing users to express domain constraints declaratively in logic, which are then 
automatically translated into well-formed English question-answer pairs. This eliminates 
the need for manual output verification, though a key challenge remains in ensuring 
semantic and grammatical correctness in the generated language—a challenge we address 
by initially constraining generation to a well-defined subset of English.

\section{Approach}

To address our research question, we developed a Prolog-based system that enables users to specify domain-specific 
constraints and generate a large number of question-answer (QA) pairs with correct answers. We implemented two versions 
of the system, each differing in the interface for specifying constraints, in order to explore trade-offs between 
expressiveness and usability.

System~1 emphasizes a minimal yet expressive interface, enabling users to generate QA pairs across a variety of 
fact-based domains with concise specifications. In contrast, System~2 offers a more verbose interface that supports 
richer natural language constructs, such as verb tense and more complex sentence structures.

\subsection{Implementation Framework}

Both systems share a common implementation framework, which consists of four major components:

\paragraph{1. Extensible Type Hierarchy} 
The first step involves defining an extensible English-type hierarchy used to annotate variables in standard 
Prolog relations, which encode the logical structure of a domain. This hierarchy must support user-defined types to 
adapt to different domains. For instance, in System~1, the base types include \texttt{person}, \texttt{location}, 
\texttt{date}, \texttt{number}, and \texttt{object}. Users can extend these using the \texttt{type(Type1, Type2)} 
relation, which asserts that \texttt{Type1} is a subtype of \texttt{Type2}. For example, \texttt{type(food, object)} 
declares that \texttt{food} should be treated as a kind of \texttt{object}.

\paragraph{2. Argument Type Annotation} 
The second step allows users to annotate the type of each argument in a given Prolog relation. In System~1, 
this is handled through the \texttt{arg/2} predicate, which takes the form \texttt{arg(RelationName, [Type1, Type2, ...])}. 
For example, if the relation \texttt{likes\_food(X, Y)} expresses that person \texttt{X} likes food \texttt{Y}, it would 
be annotated as \texttt{arg(likes\_food, [person, food])}, specifying the expected types for each argument.

\paragraph{3. English Meta-Template Mapping} 
Next, we define a set of high-level English metatemplates that describe how Prolog relations are mapped to natural 
language QA formats. Each template is associated with a specific argument signature. In System~1, we provide three 
such templates:

\begin{itemize}
    \item \textbf{Canonical:} Standard Wh-questions such as ``What is your name?'' Applicable to binary relations 
    where the first argument is of any base type and the second argument is of type \texttt{object}.
    \item \textbf{Possessive:} Questions of the form ``What is Nevada's capital?'', also for binary relations 
    where the first argument is of any base type and the second argument is of type \texttt{object}.
    \item \textbf{Canonical Negative:} Negative Wh-questions like ``What states border California but not Nevada?'', 
    designed for ternary relations where all arguments are of type \texttt{object}.
\end{itemize}

Each metatemplate defines how arguments of specific relations should be rendered into grammatically correct 
question-answer pairs. 
The \texttt{predicate\_bucket} relation \texttt{predicate\_bucket(pred, metatemplate)} is used 
to annotate the metatemplate that each predicate in the logic program should be mapped to.

\paragraph{4. Natural Language Enrichment} 
The final step supports additional annotations to enhance the fluency and grammaticality of the generated language. 
In System~1, this includes:

\begin{itemize}
    \item \textbf{Verb annotations}, such as \texttt{verb(likes\_food, ["like", "likes"])}, to indicate the 
    correct singular and plural forms of the verb for the \texttt{likes\_food} relation.
    \item \textbf{Modifier annotations}, which allow the inclusion of descriptive phrases, specified in 
    singular and plural forms to enrich sentence structure.
\end{itemize}

\subsection{End-to-End Flow}

Together, these annotations form a complete specification that allows the system to automatically generate QA pairs. 
The system leverages Prolog’s unification engine to enumerate all valid facts that satisfy the annotated relations. 
These are then translated into natural language using the metatemplates and annotations provided by the user.

An example of a complete annotation file for System~1 is shown below:

\begin{quote}
\captionof{lstlisting}{annotations.pl}
\begin{lstlisting}[language=Prolog]
generate_predicates([
    abbrev, simple_border, complex_logic_border, riverflow, 
    statecapital, population, stateriverflow, statemajorcity
]).

type(myriver, object).
type(mystate, object).

arg(simple_border, [mystate, mystate]).
arg(riverflow, [myriver, mystate]).
arg(stateriverflow, [mystate, myriver]).
arg(statemajorcity, [mystate, object]).
arg(complex_logic_border, [mystate, mystate, mystate]).
arg(population, [mystate, object]).
arg(abbrev, [mystate, object]).
arg(statecapital, [mystate, object]).

verb(simple_border, ["border", "borders"]).
verb(riverflow, ["flow through", "flows through"]).
verb(stateriverflow, ["contain", "contains"]).
verb(statemajorcity, ["contain", "contains"]).
verb(complex_logic_border, ["border", "borders"]).
verb(population, ["population", "population"]).
verb(abbrev, ["abbreviation", "abbreviation"]).
verb(statecapital, ["capital", "capital"]).

modifier(simple_border, ["state", "states"]).
modifier(complex_logic_border, ["state", "states"]).
modifier(riverflow, ["state", "states"]).
modifier(stateriverflow, ["river", "rivers"]).
modifier(statemajorcity, ["major city", "major cities"]).

predicate_bucket(simple_border, cannonical).
predicate_bucket(riverflow, cannonical).
predicate_bucket(stateriverflow, cannonical).
predicate_bucket(statemajorcity, cannonical).
predicate_bucket(complex_logic_border, cannonical_negative).

predicate_bucket(population, possessive).
predicate_bucket(abbrev, possessive).
predicate_bucket(statecapital, possessive).
\end{lstlisting}
\end{quote}

Using this framework, the system can systematically and scalably generate high-quality, 
grammatically correct QA pairs suitable for evaluating the domain-specific performance of large language models.

\section{Experiment Evaluation}

\subsection{Experimental Setup}

The primary objective of our experiments is to evaluate the degree to which our generated prompts 
resemble human-written prompts and to assess their effectiveness in verifying the behavior of a language model. 
To investigate this, we conducted a two-part evaluation framework.

As a baseline, we selected the \texttt{Geobase} dataset, which consists of U.S. geography facts represented in Prolog, 
alongside manually authored question–answer pairs. This dataset is well-suited to our goals, as it covers a complex, 
logic-based domain that aligns with the nature of our system. The human-written questions in \texttt{Geobase} serve as 
a reference point, enabling us to measure how accurately and naturally our generated prompts reflect the intended 
queries.

For the language model under evaluation, we chose LLaMA 2 (7B), a 7-billion-parameter model trained on 2 trillion 
tokens and fine-tuned using over 1 million human annotations. This model offers a practical balance between 
scalability and output quality, making it an appropriate choice for evaluating a large volume of prompts.

Part 1 of our evaluation compares the generated prompts to similar ones from the \texttt{Geobase} dataset, 
using BLEU and BERTScore to measure both syntactic and semantic similarity. To ensure a fair and meaningful 
comparison, we selected \texttt{Geobase} prompts across various categories that closely aligned in meaning with 
prompts from our dataset.

Part 2 focuses on evaluating the response quality of the LLaMA 2 model when prompted with our generated questions. 
The model's answers are compared against ground-truth responses from our dataset using accuracy, BLEU, and BERTScore. 
This evaluation is designed to measure how effectively our prompts elicit correct and semantically appropriate 
responses.  

For the accuracy metrics, we performed a direct string match between the model’s output and the ground-truth answer. 
A response was considered correct only if it matched exactly.

For the similarity metrics, we computed:
\begin{itemize}
    \item BLEU scores to assess surface-level overlap between generated and reference answers.
    \item BERT precision, recall, and F1 to evaluate semantic similarity, capturing deeper meaning even 
    when the wording differs.
\end{itemize}

This multi-metric approach allows us to assess both the literal correctness and the semantic adequacy of the LLM’s 
responses, providing a more comprehensive picture of how well the generated prompts perform in practice.

In total, over 400 prompts in varying degrees of complexity, sampled from a pool of 30,000 generated prompts, 
were used to evaluate the language model. We ran these evaluations both with and without system prompts to see 
whether the system-level instructions we used to shape the model’s responses introduced any unintended bias, 
particularly in cases where we needed outputs to align with strict string-matching criteria.

\subsection{Experimental Results and Analysis}

\subsubsection{Part 1: Structural and Semantic Similarity Evaluation}

In Table~\ref{tab:bleu_bert}, we present the BLEU and BERTScore results comparing our generated prompts to 
those in the baseline dataset, \texttt{Geobase}. Prompt pairs used for this comparison are 
shown in Figure~\ref{fig:paired_prompts}.

\begin{table}[H]
\centering
\caption{BLEU and BERTScore results}
\label{tab:bleu_bert}
\csvreader[
    tabular=lcccc,
    table head=\toprule Category & BLEU Score & BERT Precision & BERT Recall & BERT F1 \\ \midrule,
    late after line=\\,
    table foot=\bottomrule,
]{figures/bleu_experiment.csv}{}%
{%
    \csvcoli & \csvcolii & \csvcoliii & \csvcoliv & \csvcolv
}
\end{table}

\begin{figure}[H]
    \centering
    \lstinputlisting[language=json]{figures/bleu_data_one_space.json}
    \caption{Paired prompts from \texttt{bleu\_experiment/data.json}}
    \label{fig:paired_prompts}
\end{figure}

As shown in the figure, the relatively low BLEU scores suggest that our system’s prompts often differ 
from human-written questions in surface structure and word choice. This is expected, as BLEU focuses on n-gram 
overlap and penalizes phrasing variation. In contrast, the high BERTScore indicates a strong degree of semantic 
similarity between the generated prompts and the baseline.

For example, the \texttt{Geobase} question \emph{``How many citizens in Alabama?''} and the system-generated prompt 
\emph{``What is Alabama's population?''} convey the same intent despite differing vocabulary. This illustrates the 
system’s ability to produce meaning-preserving prompts even with varied phrasing.

Overall, these results suggest that while our system may not always replicate the syntactic style of human-authored 
prompts, it is effective at generating questions that retain the underlying semantics.

\subsubsection{Part 2: Verifying LLM Performance}

Tables~\ref{tab:no_sys_prompt} and \ref{tab:with_sys_prompt} present evaluation results of the LLaMA 2 (7B) 
model when prompted without and with manually engineered system-level instructions, respectively.

\begin{table}[H]
\centering
\caption{Llama Evaluation Results with \textbf{No} System Prompts}
\label{tab:no_sys_prompt}
\resizebox{\textwidth}{!}{%
\begin{tabular}{lrrrrrrrr}
\toprule
\textbf{Category} & \textbf{Q's Gen} & \textbf{Q's Asked} & \textbf{Correct} & \textbf{Passed} & \textbf{BLEU} & \textbf{BERT P.} & \textbf{BERT R.} & \textbf{BERT F1} \\
\midrule
Abbreviation Knowledge & 51 & 51 & 46 & 90.2\% & 93.10 & 0.99 & 1.00 & 0.99 \\
Capital Knowledge & 51 & 51 & 18 & 35.3\% & 51.39 & 0.95 & 0.96 & 0.96 \\
Rivers Knowledge & 46 & 46 & 0 & 0.0\% & 3.59 & 0.81 & 0.91 & 0.86 \\
State Simple Border Questions & 49 & 49 & 0 & 0.0\% & 3.25 & 0.83 & 0.92 & 0.87 \\
States Complex Border and Location questions AND & 543 & 100 & 0 & 0.0\% & 25.08 & 0.90 & 0.95 & 0.92 \\
States Complex Border Questions AND BUT & 23601 & 100 & 0 & 0.0\% & 12.84 & 0.84 & 0.91 & 0.87 \\
States Complex Total population & 19600 & 100 & 0 & 0.0\% & 4.77 & 0.82 & 0.89 & 0.85 \\
\bottomrule
\end{tabular}%
}
\end{table}

\begin{table}[H]
\centering
\caption{Llama Evaluation Results with System Prompts}
\label{tab:with_sys_prompt}
\resizebox{\textwidth}{!}{%
\begin{tabular}{lrrrrrrrr}
\toprule
\textbf{Category} & \textbf{Q's Gen} & \textbf{Q's Asked} & \textbf{Correct} & \textbf{Passed} & \textbf{BLEU} & \textbf{BERT P.} & \textbf{BERT R.} & \textbf{BERT F1} \\
\midrule
State Abbreviation Knowledge & 51 & 51 & 50 & 98.0\% & 98.39 & 1.00 & 1.00 & 1.00 \\
State Capital Knowledge & 51 & 51 & 48 & 94.1\% & 95.56 & 0.99 & 0.99 & 0.99 \\
Rivers Knowledge & 46 & 46 & 4 & 8.7\% & 54.09 & 0.96 & 0.98 & 0.97 \\
State Simple Border Questions & 49 & 49 & 3 & 6.1\% & 45.05 & 0.97 & 0.97 & 0.97 \\
States Complex Border and Location questions AND & 543 & 100 & 37 & 37.0\% & 74.21 & 0.97 & 0.98 & 0.98 \\
States Complex Border Questions AND BUT & 23601 & 100 & 2 & 2.0\% & 43.34 & 0.95 & 0.95 & 0.95 \\
States Complex Total population & 19600 & 100 & 0 & 0.0\% & 29.93 & 0.89 & 0.91 & 0.90 \\
\bottomrule
\end{tabular}%
}
\end{table}

\begin{figure}[H]
\centering
\caption{Inaccuracies ellicited by Generated Prompts}
\label{fig:no_sys_prompt_example}
\begin{tcolorbox}[colback=gray!5!white, colframe=gray!50!black, title=, fonttitle=\bfseries, breakable, width=\textwidth]
\textbf{Question:} Which state do Illinois, Ohio, and Virginia neighbor in common?

\medskip
\textbf{LLM Answer:} Illinois, Ohio, and Virginia all border the state of Pennsylvania. Therefore, they share a common border with Pennsylvania.

\medskip
\textbf{Correct Answer:} Illinois, Ohio, and Virginia neighbor Kentucky in common.
\end{tcolorbox}
\end{figure}

\begin{figure}[H]
\centering
\caption{No System Prompt, Wrong Answer}
\begin{tcolorbox}[colback=gray!5!white, colframe=gray!50!black, title=, fonttitle=\bfseries, breakable, width=\textwidth]
\textbf{Question:} What is district of columbia's capital?

\medskip
\textbf{LLM Answer:} \emph{The District of Columbia (DC) does not have a capital...}

\medskip
\textbf{Correct Answer:} District of Columbia's capital is Washington.
\end{tcolorbox}
\end{figure}

% Example A
\begin{figure}[H]
\centering
\caption{System Prompt, Right Answer}
\label{fig:example_sys_prompt_help}
\begin{tcolorbox}[colback=gray!5!white, colframe=gray!50!black, title=, fonttitle=\bfseries, breakable, width=\textwidth]
\textbf{Question:} What is district of columbia's capital?

\medskip
\textbf{LLM Answer:} \emph{Washington D.C.'s capital is Washington.}

\medskip
\textbf{Correct Answer:} District of Columbia's capital is Washington.
\end{tcolorbox}
\end{figure}

\newpage
% Example B
\begin{figure}[H]
\centering
\caption{High Similarity, Factual Error}
\label{fig:example_high_similarity_error}
\begin{tcolorbox}[colback=gray!5!white, colframe=gray!50!black, title=, fonttitle=\bfseries, breakable, width=\textwidth]
\textbf{Question:} What states does California border?

\medskip
\textbf{LLM Answer:} \emph{California borders Arizona, Nevada, Oregon, and Mexico.}

\medskip
\textbf{Correct Answer:} California borders Arizona, Nevada, and Oregon.

\medskip
\textbf{Scores:} BLEU = 53.48; BERT Precision = 0.97, Recall = 0.99, F1 = 0.98
\end{tcolorbox}
\end{figure}


We observed a clear trend in both evaluations: accuracy, precision, recall, and F1 scores declined as the question 
complexity increased. Regardless of prompt condition, the model occasionally made factual errors, as illustrated in 
Figure~\ref{fig:no_sys_prompt_example}. This supports the value of our prompts in identifying model weaknesses and 
guiding model verification.

Though system prompts generally improved accuracy, they sometimes introduced bias. For instance, 
in the category \emph{Geography: U.S. State Abbreviation Knowledge}, all responses marked incorrect 
without system prompts were semantically correct but differed in phrasing. Consequently, the real accuracy 
in this category is effectively 100\% without system prompts. In contrast, one actual factual error occurred 
with system prompts, yielding a real accuracy of 98.0\%.



In the case shown in Figure~\ref{fig:example_high_similarity_error}, the LLM mistakenly included \emph{Mexico} as a U.S. state, yet the similarity metrics remained high. 
This reveals a limitation of BLEU and BERTScore: they may signal high agreement even when responses contain 
factual inaccuracies.

\paragraph{Dynamic Information and Accuracy}

Questions involving dynamic information—such as population—proved especially difficult. 
This category scored 0\% accuracy across both runs. Since population is time-sensitive and can differ 
by source, multiple plausible answers exist, making strict string matching unsuitable. This result 
emphasizes the importance of designing constraints that elicit verifiable, concrete answers when 
using string-based evaluation.

\section{Conclusion}

In this work, we presented a Prolog-based framework designed to generate large sets of domain-specific question–answer 
pairs by allowing users to specify precise constraints. By implementing two system versions with different interfaces, 
we explored the balance between expressiveness and ease of use in prompt generation.

Our evaluation showed that the prompts generated by our system, while structurally distinct from human-written examples, 
closely match their semantic content, as evidenced by high BERTScores despite lower BLEU scores. This indicates that 
our approach effectively captures the intended meaning of complex queries even if phrasing varies.

Testing the LLaMA 2 (7B) model with these prompts demonstrated the system’s strength in detecting factual errors, 
especially as question complexity increased. We observed that system prompts generally improved accuracy but could 
sometimes introduce bias, underscoring the need for careful prompt design. Additionally, the limitations of current 
automated evaluation metrics, such as BERTScore and BLEU, became apparent, particularly in distinguishing factual 
correctness.

Overall, our approach offers a powerful tool for generating meaningful, constraint-driven prompts that can rigorously 
verify large language models in logic-based domains. Future work may focus on expanding the system to generate more 
logically complex prompts, along with stronger evaluation metrics.

\newpage
{
\small

\bibliography{references}
\bibliographystyle{plainnat}
}


\end{document}